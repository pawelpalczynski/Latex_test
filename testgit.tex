\documentclass[12pt, a4paper]{article}
\usepackage[utf8]{inputenc}
\usepackage{amsmath}
\usepackage{amsfonts}
\usepackage{amssymb}
\usepackage{graphicx}
\usepackage[left=2cm,right=2cm,top=2cm,bottom=2cm]{geometry}
\usepackage{wrapfig}
\usepackage{subfigure}
\usepackage{blindtext}
\usepackage{tabularx}
\usepackage{epsfig}

\begin{document}

\section{Introduction}

\subsection{DSC and TGA}
The typical glass used as bio-active component is composed of Si, Al, Ca, P, O, and F. As glass it has no long range order and therefore is not a crystal. However if heated it will eventually crystallise. This phase change is exothermic and will result in difference in heat flux between the sample and the reference if measured using Differential Scanning Calorimetry. In this technique the sample and the reference are heated at the same rate and the heat flux is measured. The heat capacity of the reference is known and it has no phase transitions in the range of temperatures used in an experiment. Any phase transition of sample results in difference in heat flux between sample and reference which allows to find the enthalpy of reactions that sample undergoes. At the same time the mass of the sample can be measured by Thermo-Gravimetric Analysis. Both of those techniques can be used together using Simultaneous Thermal Analyser. During crystallisation no mass change is expected. The heat of reaction can be usually calculated from the area of crystallisation peak using equation \ref{eq:EnthalpyOfReaction}: 

\begin{equation}
Area = \frac{\Delta H m}{K}
\label{eq:EnthalpyOfReaction}
\end{equation} 

\end{document}